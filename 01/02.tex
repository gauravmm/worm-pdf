





My thoughts were on Emma on the bus ride home.  For an outside observer, I think it's easy to trivialize the importance of a `best friend', but when you're a kid, there's nobody more important.  Emma had been my `BFF' from grade one all the way through middle school.  It hadn't been enough for us to spend our time together at school, so we had alternated staying at each others houses every weekend.  I remember my mother saying that we were so close we were practically sisters.



A friendship that deep is intimate.  Not in the rude way, but just in terms of a no-holds-barred sharing of every vulnerability and weakness.



So when I got back from nature camp just a week before our first year at high school started, to find that she wasn't talking to me?  That she was calling Sophia her best friend?  Discovering that she was now using every one of those secrets and vulnerabilities I had shared with her to wound me in the most vicious ways she could think of?  It was crushing.  There's just no better way to say it.



Unwilling to dwell on it any longer, I turned my attention to my backpack, setting it on the seat beside me and sorting through the contents.  Grape juice had stained it, and I had a suspicion I would have to get a new one.  I had bought it just four months ago, after my old one had been taken from my locker, and it had been just twelve bucks, so it wasn't a huge issue.  The fact that my notebooks, textbooks and the two novels I'd shoved into my bag were wet with grape juice was more troubling.  I suspected that whichever girl had been holding the grape juice had aimed for the open top of my bag as she poured it.  I noted the destruction of my art project – the box I'd put it in was collapsed on the one side.  That bit was my fault.



My heart sank as I found the notebook with the white and black speckled hardcover.  The corner of the paper was soaked through with as much as a quarter of each page stained purple.  The ink had diluted and the pages were already turning wavy.



That notebook was – had been – my notes and journal for my hero career.  The testing and training I'd done with my powers, pages of crossed out name ideas, even the measurements I was using for my costume in progress.  After Emma, Madison and Sophia had stolen my last backpack and stuffed it in a wastebasket, I had realized how big a danger it was to have everything written down like that.  I had copied everything over into a new notebook in a simple cipher and wrote it bottom to top.  Now that notebook was spoiled, and I was looking at having to copy some two hundred pages of detailed writing into a new notebook if I wanted to preserve the information.  If I could even remember what was on all of the ruined pages.



The bus stopped a block away from my house, and I got off, trying to ignore the stares.  Even with the gawking, the knowledge that my notebook was ruined and my general nervousness about missing afternoon classes without permission, I felt better as I got closer to home.  It felt worlds better to know I could drop my guard, stop watching my back and that I could take a break from wondering when the next incident would happen.  I let myself into the house and headed straight for the shower, not even removing my backpack or taking off my shoes until I was in the bathroom.



I stood under the stream with my clothes on the floor of the tub, hoping the water would help get the worst of the juice out.  I pondered.  I don't know who said it, but at one point I had come across this notion about taking a negative and turning it into a positive.  I tried to take the day's events and turn them around in my head, to see if I couldn't find a more positive twist on it.



Okay, so the first thing that came to mind was ``Yet another reason to kill the trio.''  It wasn't a serious thought – I was angry, but it wasn't like I was going to actually kill them.   Somehow, I suspected that I'd hurt myself before I hurt them.  I was humiliated, frustrated, pissed, and I always had a weapon available – my power.  It was like having a loaded gun in your hand at all times.  Except my power wasn't that great, so maybe it was more like having a taser.  It was hard not to think about using it when things got really bad.  Still, I didn't think I had that killer instinct in me.



No, I told myself, forcing myself back to the subject of positive thinking.  Were there any upsides?  Art project wrecked, clothes probably unrecoverable, needing a new backpack\ldots  notebook.  Somehow my mind fixated on that last part.



I cranked the shower to off, then toweled dry, thinking.  I wrapped the towel around me, and rather than head to my room to get dressed, I put my wet clothes into a laundry hamper, grabbed my backpack and headed downstairs, through the kitchen and into the basement.



My house is old, and the basement was never renovated.  The walls and floor are concrete and the ceiling was exposed boards and electrical cords.  The furnace used to be coal fueled, and there was still an old coal chute, two feet by two feet, where the coal trucks used to come by to unload the winter's supply of coal for heating the house.  The chute was boarded up, but around the time I was copying my original `superpower notebook' over in code, I had decided to play it safe in all respects and start getting creative with my privacy.  It was then that I'd started using it.



I removed one screw and removed the square wooden panel with the peeling white paint that covered the low end of the coal chute.  I retrieved a gym bag from inside and put the panel back in place without screwing it back in.



I emptied the contents of the gym bag on the disused workbench that the house's previous owner had left in our basement, then opened the windows that were at the same level as the driveway and front garden.  I closed my eyes and spent a minute exercising my power.  I wasn't just grabbing every creepy crawly in a two block radius, though.  I was being selective, and I was gathering quite a few.



It would take time for all of them to arrive.  Bugs could move faster than you thought when they moved with purpose in a straight line, but even so, two blocks was a lot of ground for something so small to cover.  I busied myself with opening the bag and sorting out the contents.  My costume.



The first of the spiders started coming in through the open windows and congregating on the workbench.  My power didn't give me a knowledge of the official names of the bugs I was working with, but anyone could recognize the spiders that were crawling into the room.  These were black widows.  One of the more dangerous spiders you could find in the States.  Their bite could be lethal, though it usually wasn't, and they tended to bite with little provocation.  Even under my complete control, they spooked me.  At my request, the dozens upon dozens of spiders got into place on the workbench and began drawing out lines of webbing, laying the lines across one another, and weaving them into one work.



Three months ago, after I'd recovered from the manifestation of my powers, I had started to prepare for the goal I had set for myself.  It had involved an exercise routine, training my power, research, and preparing my costume.  Costumes were harder than one might think.  While members of official teams surely had sources for that stuff, the rest of us were left to either buy costumes, put them together piecemeal with stuff bought from stores or make them from scratch.  Each option had its problems.  If you bought a costume online, you ran the risk of being traced, which could blow your secret identity before you'd even put a costume on.  You could put a costume together with stuff bought from stores, but very few people could do that and look good.  The final option, putting a costume together yourself, was just a hell of a lot of work and you could run into the issues of the prior two options – being traced or winding up with a lame costume – depending on where you got your materials and how you went about it.



In the second week after I'd figured out my powers, when I still wasn't entirely sure what was going on, I had come across a segment on the discovery channel about a suit that was made to withstand attacks by bears.  That segment talked about how the suit was made of synthetic spider silk, which had inspired this particular project.  Why go synthetic when you can produce with the real thing?



Okay, so it had been harder than that.  Not just any spider worked, and the black widow spiders themselves were hard to find.  They weren't typically found in the northeastern states, where it was generally colder, but I was fortunate that that key element that made Brockton Bay a tourist destination and a hotspot for capes also made it a place where black widow spiders could live, if not thrive.  Namely, it was warm.  Thanks to the surrounding geography and the ocean bordering us on the east, Brockton Bay had some of the mildest winters you could find in the Northeastern States, and some of the most comfortably warm summers.  Both the black widows and the people running around in skintight costumes were thankful for that.



With my power, I had ensured the spiders could multiply.  I'd kept them in safe locations and fattened them on prey I directed straight to them.  I had flipped that mental switch that told them to breed and lay eggs as if it was summer, fed more prey to the hundreds of young that had resulted and had earned countless costume spinners for my trouble.  The biggest issue had been that black widows are territorial, so I'd had to spread them out to ensure they didn't kill each other when I wasn't around to control them.  Once a week or so, on my morning runs, I rotated the locations of the local spiders so I had a fresh supply all filled with proteins for the production of the essential materials.  This ensured that the spiders were always ready for working on the costume in the afternoon, after school.



Yeah, I needed a life.



But I had a badass costume.



It wasn't a great looking costume, just yet.  The fabric was a dirty yellow-gray.  The armored sections had been made out of finely arranged and layered shells and exoskeletons I'd cannibalized from the local insect population and then reinforced with dragline silk.  In the end, the armored parts had wound up dark mottled brown-gray.  I was okay with that.  When the entire thing was done, I planned to dye the fabric and paint the armor.



The reason I was so pleased with my costume was the fact that it was flexible, durable, and incredibly lightweight, considering the amount of armor I had put on it.  At one point I had screwed up the dimensions of one of the legs, and when I tried to cut it off to start fresh, I had found I couldn't cut it with an x-acto knife.  I had needed to use wire cutters, and even that had been a chore.  As far as I figured, it was everything a superhero wanted for a costume.



I wasn't exactly willing to test it out, but I harbored hopes that it was bulletproof.  Or at least, that the armored sections over my vital areas were.



The plan was to finish my costume over the course of the month, then as the school year ended and the summer began, I would take the leap into the world of superheroics.



But the plan had changed.  I took off my towel and hung it from the corner of the bench, then began pulling on my costume to test the fit for the hundredth time.  The spiders obediently moved out of my way as I did so.



When I had been standing in the shower, trying to find the good aspects in the day's troubles, my thoughts had turned to my notebook.  I had realized I was procrastinating.  I was constantly planning, preparing, considering all of the possibilities.  There would always be more preparations, more stuff to study or test.  The destruction of my notebook had been the burning of a bridge.  I couldn't go back and copy it into a fresh book or start a new one without delaying my game plan for at least a week.  I had to move forward.



It was time to do it.  I flexed my hand inside the glove.  I'd go out next week – no.  No more delays.  This weekend, I would be ready.





