





I heard the cape arrive on his souped up motorcycle.  I didn't want to be seen fleeing the scene of a fight, and risk being labeled one of the bad guys by \emph{yet another} person, but I wasn't about to get closer to the street either, in case Lung was feeling better.  Since there was nowhere to go, I just stayed put.  Just resting felt good.



If you'd asked me just a few hours ago about how I thought I would feel meeting a big name superhero, I would have used words like excited and giddy.  The reality was that I was almost too exhausted to care.



It looked as though he flew up onto the roof, but the six-foot long weapon the man held kind of jerked as he landed.  I was pretty sure I saw the tines of a grappling hook retreating back into the end of the weapon.  So this was what Armsmaster looked like in person, I thought.



The largest superhero organization in the world was the Protectorate, spanning Canada and the States, with ongoing talks about including Mexico in the deal.  It was a government sponsored league of superheroes with a base in each `cape city'.  That is, they had a team set up in each city with a sizable population of heroes and villains.  Brockton Bay's team was officially `The Protectorate East-North-East', and were headquartered in the floating, forcefield-shrouded island that you could see from the Boardwalk.  This guy, Armsmaster, was the guy in charge of the local team.  When the core group of the top Protectorate members from around Canada and the States assembled in that classic `v' formation for the photo shoots, Armsmaster was one of the guys in the wings.  This was a guy who had his own action figures.  Poseable Armsmaster with interchangeable Halberd parts.



He really did look like a superhero, not like some guy in a costume.  It was an important distinction.  He wore body armor, dark blue with silver highlights, had a sharply angled v-shaped visor covering his eyes and nose.  With only the lower half of his face exposed, I could see a beard trimmed to trace the edges of his jaw.  If I had to judge, with only the lower half of his face to go by, I'd guess he was in his late twenties or early thirties.



His trademark and weapon was his Halberd, which was basically a spear with an axe head on the end, souped up with gadgets and the kind of technology you generally only saw in science fiction.  He was the kind of guy who appeared on magazine covers and did interviews on TV, so you could find almost anything about Armsmaster through various media, short of his secret identity.  I knew his weapon could cut through steel as though it was butter, that it had plasma injectors for stuff that the blade alone couldn't cut and that he could fire off directed electromagnetic pulses to shut down forcefields and mechanical devices.



``You gonna fight me?'' He called out.



``I'm a good guy,'' I said.



Stepping closer to me, he tilted his head, ``You don't look like one.''



That stung, especially coming from him.  It was like Michael Jordan saying you sucked at basketball.  ``That's\ldots not intentional,'' I responded, not a little defensively, ``I was more than halfway done putting the costume together when I realized it was already looking more edgy than I'd intended, and I couldn't do anything about it by then.''



There was a long pause.  Nervously, I turned my eyes from that opaque visor.  I glanced at his chest emblem, a silhouette of his visor in blue against a silver background, and was struck with the ridiculous thought that I had once owned a pair of underpants with his emblem on the front.



``You're telling the truth,'' he said.  It was a definitive statement, which startled me.  I wanted to ask how he knew, but I wasn't about to do or say anything that might change his mind.



He approached closer, looking me over as I sat there with my arms around my knees, he asked, ``You need a hospital?''



``No,'' I said. ``Don't think so.  I'm as surprised as you are.''



``You're a new face,'' he said.



``I haven't even come up with a name yet.  You know how hard it is to come up with a bug-themed name that doesn't make me sound like a supervillain or a complete dork?''



He chuckled, and it sounded warm, very normal, ``I wouldn't know.  I got into the game early enough that I didn't have to worry about missing out on all of the good names.''



There was a pause in the conversation.  I suddenly felt awkward.  I don't know why, but I admitted to him, ``I almost died.''



``That's why we have the Ward program,'' he said.  There was no judgement in his tone, no pressure.  Just a statement.



I nodded, more to give a response than out of any agreement with the answer.  The Wards were the under-eighteen subdivision of the Protectorate, and Brockton Bay did have its own team of Wards, with the same naming convention as the Protectorate; The Wards East-North-East.  I had considered applying to join, but the notion of escaping the stresses of high school by flinging myself into a mess of teenage drama, adult oversight and schedules seemed self-defeating.



``You get Lung?'' I asked, to change the subject from the Wards.  I was pretty sure that he was obligated to try and induct new heroes into either the Protectorate or the Wards, depending on their age, to promote the whole agenda of organized heroes who are accountable for their actions, and I really didn't want him to get on my case about joining.



``Lung was unconscious, beaten and battered when I arrived.  I pumped him full of tranquilizers to be safe and temporarily restrained him under a steel cage I welded to the sidewalk.  I'll pick him up on my way back.''



``Good,'' I said, ``With him in jail, I'll feel like I accomplished something today.  Only reason I started the fight was because I overheard him telling his men to shoot some kids.  Only realized later that he was talking about some other villains.''



Armsmaster turned to look at me.  So I told him, walking him through the fight in general, the arrival of the teenage bad guys, and their general descriptions.  Before I finished, he was pacing back and forth on the roof.



``These guys.  They knew I was coming?''



I nodded, once.  As much respect as I had for Armsmaster, I wasn't in much of a mood to repeat myself.



``That explains a lot,'' he said, staring off into the distance.  After a few moments, he went on to explain, ``They're slippery.  On those few occasions we do manage to get in a toe to toe fight with them, they either win, or they get away more or less unscathed, or both.  We know so little about them.  Grue and Hellhound were working on their own before they joined the group, so there's some information there, but the other two?  They're nonentities.  If the girl Tattletale has some way of detecting or tracking us, it would go a long way towards explaining why they're doing as well as they are.''



It kind of surprised me to hear one of the top level heroes admitting to being anything less than perfectly on top of things.



``It's funny,'' I said, after a few moment's thought, ``They didn't seem that hardcore.  Grue said they were kind of panicking when they heard Lung was coming after them, and they were casually joking around while the fight was going on.  Grue was making fun of Regent.''



``They said all this in front of you?'' he asked.



I shrugged, ``I think they thought I was helping them out.  The way Tattletale talked, I think she thought I was a bad guy too or something.''  With a touch of bitterness, I said, ``Dunno, I guess it was the costume that led them to that assumption.''



``Could you have taken them in a fight?'' Armsmaster asked me.



I started to shrug, and winced a little.  I was feeling a little sore in the shoulder, where I'd tumbled on the roof after being blasted by Lung's flames.  I said, ``Like you said, we don't know a lot about them, but I think that girl with the dogs--''



``Hellhound,'' Armsmaster said.



``I think she could have kicked my ass on her own, so no.  I probably couldn't have fought them.''



``Then count it as a good thing that they got the wrong impression,'' Armsmaster said.



``I'll try to look at it that way,'' I said, struck by how he easily he was able to employ the whole `take a negative and turn it into a positive' mindset I'd been trying to maintain.  I envied that.



``That a girl,'' he said, ``And while we're looking forward, we need to decide where we go from here.''



My heart sank.  I knew he was going to bring up the Wards again.



``Who gets the credit for Lung?''



Caught off guard, I looked up at him.  I started to speak, but he held up his hand.



``Hear me out.  What you've done tonight is spectacular.  You played a part in getting a major villain into custody.  You just need to consider the consequences.''



``Consequences,'' I muttered, even as the word \emph{spectacular} rang in my ears.



``Lung has an extensive gang throughout Brockton Bay and neighboring cities.  More than that, he has two superpowered flunkies.  Oni Lee and Bakuda.''



I shook my head, ``I know about Oni Lee, and Grue mentioned fighting him.  I've never heard of Bakuda.''



Armsmaster nodded, ``Not surprising.  She's new.  What we know about her is limited.  She made her first appearance and demonstration of her powers by way of a drawn out terrorism campaign against Cornell University.  Lung apparently recruited her and brought her to Brockton Bay after her plans were foiled by the New York Protectorate.  This is\ldots something of a concern.''



``What are her powers?''



``Are you aware of the Tinker classification?''



I started to shrug, but remembered my sore shoulder and nodded instead.  It was probably more polite, too.  I said, ``Covers anyone with powers that give them an advanced grasp of science.  Lets them make technology years ahead of its time.  Ray guns, ice blasters, mechanized suits of armor, advanced computers.''



``Close enough,'' Armsmaster said.  It struck me he would be a Tinker, if his Halberd and armor were any indication.  That, or he got his stuff from someone else.  He elaborated,  ``Well, most Tinkers have a specialty or a special trick.  Something they're particualrly good at or something that they can do, which other Tinkers can't.  Bakuda's specialty is bombs.''



I stared at him.  A woman with a power that let her make bombs that were technologically decades ahead of their time.  No wonder he saw it as a concern.



``Now I want you to consider the danger involved in taking the credit for Lung's capture.  Without a doubt, Oni Lee and Bakuda will be looking to accomplish two goals.  Freeing their boss and getting vengeance on the one responsible.  I suspect you're now aware\ldots these are scary people.  Scarier in some ways than their boss.''



``You're saying I shouldn't take the credit,'' I said.



``I'm saying you have two options.  Option one is to join the Wards, where you'll have support and protection in the event of an altercation.  Option two is to keep your head down.  Don't take the credit.  Fly under the radar.''



I wasn't prepared to make a decision like that.  Usually, I went to sleep at eleven or so, waking up at six thirty to get ready for my morning run.  At my best guess, it was somewhere between one and two in the morning.  I was emotionally exhausted from the highs and lows of the evening, and I could barely wrap my head around the complications and headaches that would come from joining the Wards, let alone having two insanely dangerous sociopaths coming after me.



On top of that, I wasn't so ignorant as to miss Armsmaster's motives.  If I opted to not take the credit for Lung's capture, Armsmaster would, I was sure.  I didn't want to get on the bad side of a major player.



``Please keep my involvement in Lung's capture secret,'' I told him, painfully disappointed to have to say it, even as I knew it made the most sense.



He smiled, which I hadn't expected.  He had a nice smile.  It made me think that he could win the hearts of a lot of women, whatever the top two-thirds of his face looked like.  ``I think you'll look back and see this was a smart decision,'' Armsmaster said, turning to walk to the other end of the roof, ``Call me at the PHQ if you're ever in a pinch.''  He stepped off the edge of the roof and dropped out of sight.



Call me if you're ever in a pinch.  He'd been saying, without openly admitting, that he owed me one.  He would take the lion's share of the credit for Lung's capture, but he owed me one.



Before I was all the way down the fire escape, I heard the thrum of his motorcycle, presumably carrying Lung towards a life of confinement.  I could hope.



It would take me a half hour to get home.  On the way, I would stop and pull on the sweatshirt and jeans I had hidden.  I knew my dad went to sleep even earlier than I did, and he slept like a log, so I had nothing to worry about as far as wrapping up the night.



It could have gone worse.  Strange as it sounds, those words were a  security blanket I wrapped around myself to keep myself from dwelling on the fact that tomorrow was a school day.





