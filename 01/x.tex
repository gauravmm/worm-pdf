





``We don't know how long he had been there.  Suspended in the air above the Atlantic Ocean.  On May twentieth, 1982, an ocean liner was crossing from Plymouth to Boston when a passenger spotted him.  He was naked, his arms to his sides, his long hair blowing in the wind as he stood in the sky, nearly a hundred feet above the gently cresting waves.  His skin and hair can only be described as a burnished gold.  With neither body hair nor clothes to cover him, it is said, he seemed almost artificial.



``After a discussion including passenger and crew, the liner detoured to get closer.  It was a sunny day, and passengers crowded to the railings to get a better look.  As if sharing their curiosity, the figure drew closer as well.  His expression was unchanging, but witnesses at the scene reported that he appeared deeply sad.



``I thought he was going to crack his facade and cry any moment', said Grace Lands, `But when I reached out and touched his fingertips, I was the one who burst into tears.'



``That boat trip was a final journey for me.  I had cancer, and I wasn't brave enough to face it.  Can't believe I'm admitting this in front of a camera, but I was going back to Boston, where I was born, to end things myself.  After I met him, I changed my mind.  Didn't matter anyways.  I went to a doctor, and he said there was no sign I ever had the disease.'



``My brother, Andrew Hawke, was the last passenger to make any sort of contact with him, I remember.  He climbed up onto the railing, and, almost falling off, he clasped the hand of the golden man.  The rest of us had to grab onto him to keep him from falling.  Whatever happened left him with a quiet awe.  When the man with the golden skin flew away, my brother stayed silent.  The rest of the way to Boston, my brother didn't say a word.  When we docked, and the spell finally broke, my brother babbled his excitement to reporters like a child.'



``The golden man would reappear several more times in the coming months and years.  At some point, he donned clothing.  At first, a sheet worn over one shoulder and pinned at either side of the waist, then more conventional clothes.  In 1999, he donned the white bodysuit he still wears today.  For more than a decade, we have wondered, where did our golden man get these things?  Who was he in contact with?



``Periodically at first, then with an increasing frequency, the golden man started to intervene in times of crisis.  For events as small as a car accident, as great as natural disasters, he has arrived and used his abilities to save us.  A flash of light to freeze water reinforcing a levee stressed by a hurricane.  A terrorist act averted.  A serial murderer caught.  A volcano quelled.  Miracles, it was said.



``His pace increased, perhaps because he was still learning what he could do, perhaps because he was getting a greater sense of where he was needed.  By the middle of the 1990s, he was traveling from crisis to crisis, flying faster than the speed of sound.  In fifteen years, he has not rested.



``He has been known to speak just once in thirty years.  After extinguishing widespread fire in Alexandrovsk, he paused to survey the scene and be sure no blazes remained.  A reporter spoke to him, and asked, `Kto vy?' – what are you?



``Shocking the world, caught on camera in a scene replayed innumerable times, he answered in a voice that sounded as though it might never have uttered a sound before.  Barely audible, he told her, `Scion'.



``It became the name we used for him.  Ironic, because we took a word that meant descendant, and used it to name the first of many superpowered individuals – parahumans – to appear across Earth.



``Just five years after Scion's first appearance, the superheroes emerged from the cover of rumor and secrecy to show themselves to the public.  Though the villains followed soon after, it was the heroes who shattered any illusions of the parahumans being divine figures.  In 1989, attempting to quell a riot over a basketball game in Michigan, the superhero known to the public as Vikare stepped in, only to be clubbed over the head.  He died not long after of a brain embolism.  Later, he would be revealed to be Andrew Hawke.



``The golden age of the parahumans was thus short lived.  They were not the deific figures they had appeared to be.  Parahumans were, after all, people with powers, and people are flawed at their core.  Government agencies took a firmer hand, and state--''



The television flicked off, and the screen went black, cutting the documentary off mid sentence.  Danny Hebert sighed and sat down on the bed, only to stand just a moment later and resume pacing.



It was three fifteen in the morning, and his daughter Taylor was not in her bedroom.



Danny ran his hands through his hair, which was thinned enough at the top to be closer to baldness than not.  He liked to be the first to arrive at work, watching everyone arrive, having them know he was there for them.  So he usually went to bed early; he'd turn in at ten in the evening, give or take depending on what was on TV.  Only tonight, a little past midnight, he'd been disturbed from restless sleep when he had felt rather than heard the shutting of the back door of the house, just below his bedroom.  He had checked on his daughter, and he'd found her room empty.



So he had waited for his daughter to return for three hours.



Countless times, he had glanced out the window, hoping to see Taylor coming in.



For the twentieth time, he felt the urge to ask his wife for help, for advice, for support.  But her side of the bed was empty and it had been for some time.  Daily, it seemed, he was struck by the urge to call her cell phone.  He knew it was stupid – she wouldn't pick up – and if he dwelt on that for too long, he became angry at her, which just made him feel worse.



He wondered, even as he knew the answer, why he hadn't gotten Taylor a cell phone.  Danny didn't know what his daughter was doing, what would drive her to go out at night.  She wasn't the type.  He could tell himself that most fathers felt that way about their daughters, but at the same time, he knew.  Taylor wasn't social.  She didn't go to parties, she wouldn't drink, she wasn't even that interested in champagne when they celebrated the New Year together.



Two ominous possibilities kept nagging at him, both too believable.  The first was that Taylor had gone out for fresh air, or even for a run.  She wasn't happy, especially at school, he knew, and exercise was her way of working through it.  He could see her doing it on a Sunday night, with a fresh week at school looming.  He liked that her running made her feel better about herself, that she seemed to be doing it in a reasonable, healthy way. He just hated that she had to do it here, in this neighborhood.  Because here, a skinny girl in her mid-teens was an easy target for attack.  A mugging or worse – he couldn't even articulate the worst of the possibilities in his own thoughts without feeling physically sick.  If she had gone out at eleven in the evening for a run and hadn't come back by three in the morning, then it meant something had happened.



He glanced out the window again, at that corner of the house where the pool of illumination beneath the streetlight would let him see her approaching.  Nothing.



The second possibility wasn't much better.  He knew Taylor was being bullied.  Danny had found that out in January, when his little girl had been pulled out of school and taken to the hospital.  Not the emergency room, but the psychiatric ward.  She wouldn't say by whom, but under the influence of the drugs they had given her to calm down, she had admitted she was being victimized by bullies, using the plural to give him a clue that it was a they and not a he or a she.  She hadn't mentioned it – the incident or the bullying – since.  If he pushed, she only tensed up and grew more withdrawn.  He had resigned himself to letting her reveal the details in her own time, but months had passed without any hints or clues being offered.



There was precious little Danny could do on the subject, either.  He had threatened to sue the school after his daughter had been taken to the hospital, and the school board had responded by settling, paying her hospital bills and promising they would look out for her to prevent such events from occurring in the future. It was a feeble promise made by a chronically overworked staff and it didn't do a thing to ease his worries.  His efforts to have her change schools had been stubbornly countered with rules and regulations about the maximum travel times a student was allowed to have between home and a given school.  The only other school within a reasonable distance of Taylor's place of residence was Arcadia High, and it was already desperately overcrowded with more than two hundred students on a list requesting admittance.



With all that in mind, when his daughter disappeared until the middle of the night, he couldn't shake the idea that the bullies might have lured her out with blackmail, threats or empty promises.  He only knew about the one incident, the one that had landed her in the hospital, but it had been grotesque.  It had been implied, but never elaborated on, that more had been going on.  He could imagine these boys or girls that were tormenting his daughter, egging one another on as they came up with more creative ways to humiliate or harm her.  Taylor hadn't said as much aloud, but whatever had been going on had been mean, persistent and threatening enough that Emma, Taylor's closest friend for years, had stopped spending time with her.  It galled him.



Impotent.  Danny was helpless where it counted.  There was no action he could take – his one call to the police at two in the morning had only earned him a tired explanation that the police couldn't act or look for her without something more to go on.  If his daughter was still gone after twelve hours, he'd been told, he should call them again.  All he could do was wait and pray with his heart in his throat that the phone wouldn't ring, a police officer or nurse on the other end ready to tell him what had happened to his daughter.



The slightest of vibrations in the house marked the escape of the warm air in the house to the cold outdoors, and there was a muffled whoosh as the kitchen door shut again.  Danny Hebert felt a thrill of relief coupled with abject fear.  If he went downstairs to find his daughter, would he find her hurting or hurt?  Or would his presence make things worse, her own father seeing her at her most vulnerable after humiliation at the hands of bullies?  She had told him, in every way except articulating it aloud, that she didn't want that.  She had pleaded with him, with body language and averted eye contact, unfinished sentences and things left unsaid, not to ask, not to push, not to see, when it came to the bullying.  He couldn't say why, exactly.  Home was an escape from that, he'd suspected, and if he recognized the bullying, made it a reality here, maybe she wouldn't have that relief from it.  Perhaps it was shame, that his daughter didn't want him to see her like that, didn't want to be that weak in front of him.  He really hoped that wasn't the case.



So he ran his fingers through his hair once more and sat down on the corner of the bed, elbows on his knees, hands on his head, and stared at his closed bedroom door.  His ears were peeled for the slightest clue.  The house was old, and it hadn't been a high quality building when it had been new, so the walls were thin and the structure prone to making noise at every opportunity.  There was the faintest sound of a door closing downstairs.  The bathroom?  It wouldn't be the basement door, with no reason for her to go down there, and he couldn't imagine it was a closet, because after two or three minutes, the same door opened and closed again.



After something banged on the kitchen counter, there was little but the occasional groan of floorboards.  Five or ten minutes after she had come in, there was the rhythmic creak of the stairs as she ascended.  Danny thought about clearing his throat to let her know he was awake and available should she knock on his door, but decided against it.  He was being cowardly, he thought, as if his clearing of his throat would give reality to his fears.



Her door shut carefully, almost inaudibly, with the slightest tap of door on doorframe.  Danny stood, abruptly, opening his door, ready to cross the hall and knock on her door.  To verify that his daughter was okay.



He was stopped by the smell of jam and toast.  She had made a late night snack.  It filled him with relief.  He couldn't imagine his daughter, after being mugged, tormented or humiliated, coming home to have toast with jam as a snack.  Taylor was okay, or at least, okay enough to be left alone.



He let out a shuddering sigh of relief and retreated to his room to sit on the bed.



Relief became anger.  He was angry at Taylor, for making him worry, and then not even going out of her way to let him know she was okay.  He felt a smouldering resentment towards the city, for having neighborhoods and people he couldn't trust his daughter to.  He hated the bullies that preyed on his daughter.  Underlying it all was frustration with himself.  Danny Hebert was the one person he could control in all of this, and Danny Hebert had failed to do anything that mattered.  He hadn't gotten answers, hadn't stopped the bullies, hadn't protected his daughter.  Worst of all was the idea that this might have happened before, with him simply sleeping through it rather than laying awake.



He stopped himself from walking into his daughter's room, from shouting at her and demanding answers, even if it was what he wanted, more than anything.  Where had she been, what had she been doing?  Was she hurt?  Who were these people that were tormenting her?  He knew that by confronting her and getting angry at her, he would do more harm than good, would threaten to sever any bond of trust they had forged between them.



Danny's father had been a powerful, heavyset man, and Danny hadn't gotten any of those genes.  Danny had been a nerd when the term was still young in popular culture, stick thin, awkward, short sighted, glasses, bad fashion sense.  What he had inherited was his father's famous temper.  It was quick to rise and startling in its intensity.  Unlike his father, Danny had only ever hit someone in anger twice, both times when he was much younger.  That said, just like his father, he could and would go off on tirades that would leave people shaking.  Danny had long viewed the moment he'd started to see himself as a man, an adult, to be the point in time where he had sworn to himself that he wouldn't ever lose his temper with his family.  He wouldn't pass that on to his child the way his father had to him.



He had never broken that oath with Taylor, and knowing that was what kept him contained in his room, pacing back and forth, red in the face and wanting to punch something.  While he'd never gotten angry at her, never screamed at her, he knew Taylor had seen him angry.  Once, he had been at work, talking to a mayor's aide.  The man had told Danny that the revival projects for the Docks were being cancelled and that, contrary to promises, there were to be layoffs rather than new jobs for the already beleaguered Dockworkers.  Taylor had been spending the morning in his office on the promise that they would go out for the afternoon, and had been in a position to see him fly off the handle in the worst way with the man.  Four years ago, he had lost his temper with Annette for the first time, breaking his oath to himself.  That had been the last time he had seen her.  Taylor hadn't been there to see him shouting at her mother, but he was fairly certain she'd heard some of it.  It shamed him.



The third and last time that he had lost his temper where Taylor had been in a position to know had been when she had been hospitalized following the incident in January.  He'd screamed at the school's principal, who had deserved it, and at Taylor's then-Biology teacher, who probably hadn't.  It had been bad enough that a nurse had threatened to call for a police officer, and Danny, barely mollified, had stomped from the hallway to the hospital room to find his daughter more or less conscious and wide eyed in reaction.  Danny harbored a deep fear that the reason Taylor hadn't offered any details on the bullying was out of fear he would, in blind rage, do something about it.  It made him feel sick, the notion that he might have contributed something to his daughter's self imposed isolation in how she was dealing with her problems.



It took Danny a long time to calm down, helped by telling himself over and over that Taylor was okay, that she was home, that she was safe.  It was something of a blessing that, as the anger faded, he felt drained.  He climbed into the left side of the bed, leaving the right side empty out of a habit he'd yet to break, and pulled the covers up around himself.



He would talk to Taylor in the morning.  Get an answer of some sort.



He dreamed of the ocean.





