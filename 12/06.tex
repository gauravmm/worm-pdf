





I never thought I'd be \emph{thankful} in any way that Leviathan had trashed my hometown.  Leviathan's tidal waves had shattered many of the windows and the residents had put plywood, plastic and boards up in their wake.  It meant there was less material for Shatterbird to use against us.  Countless people had been spared from injury and death due to Shatterbird's glass shards because Leviathan had gotten to us first.



But even without the glass, there was still \emph{sand}.



I stepped out of the way as a trio of people moved down the street, supporting each other as much as they were able.  Each of them had been blasted by the sand, their skin left ragged.  It had turned a bruised combination of black brown and purple where it hadn't been scraped off and left raw, red and openly bleeding.  One looked as though he'd been blinded.  The sandburns covered his upper face.



Two ambulances had stopped at an intersection just a block away from where I had announced my claim of territory.  At a glance, I could tell that they'd had all mirrors removed and all glass stripped from the dash, doors and windshield.  Those that had emerged from their homes and shelters were gravitating towards the ambulances.  There was still dust settling on the streets, and I could taste it thick in the air, even through my mask.  I wondered if we needed to be getting masks out to people.  It couldn't be healthy.



Heads turned as I approached.  I'd put my costume on again, and I had a swarm of bugs following in my wake, giving me more presence.  When people were this hurt and scared, it didn't take much to tap into that primal part of their psyches and intimidate them just a little.



Surveying the scene, I could already tell there were going to be issues.



Hundreds, \emph{thousands} of hurt people, many in critical or potentially critical shape, there were only two ambulances here, and the hospitals would be overcrowded.  People were going to panic when they realized that they wouldn't necessarily get help.  They would get upset, even angry.  This already unstable situation would descend into all-out chaos.



\emph{I told them I'd protect them, but there was no stopping this}.



I wasn't on my game.  My thoughts were on Dad and on Tattletale, not on these people and all the factors that I was supposed to take into account.  But I didn't have a choice.



I gave the order, and my swarm spread out, flowing through the crowd.  It was enough bugs to get people's attention.  I just hoped the benefits of having the bugs there would outweigh any fear or discomfort the bugs generated.



Using the bugs I'd spread around the area, I augmented my voice, allowing it to carry.  ``The most important thing is to remain calm.''



More people turned toward me.  I stepped closer to the ambulances, where paramedics were working with some of the most critical cases.  I felt like a charlatan, a pretender.  The look of mixed fear and incredulity from the paramedics didn't help.  Still, \emph{someone} had to take control and organize before people started lashing out, and the city's heroes were apparently occupied elsewhere.



``I don't intend you any harm,'' I reassured them.  ``If you're unhurt and able-bodied, there are people who need your help.  Step forward so I can direct you to them.''



Silence and stillness stretched on for long seconds.  I could \emph{see} people who had no visible injuries, who were staring at me, unwilling to respond to my appeal.  Generally speaking, the types of people who lived in the Docks weren't the sort who were used to being neighborly, to putting society's needs above their own.



Fuck me.  My head wasn't in the right place.  I'd forgotten.  I'd been taught in the first aid classes you had to be direct and specific when dealing with people in a crisis.  Asking for help was begging for disappointment, because people would hesitate to step forward, or assume that someone \emph{else} would handle the job.  Instead of asking for help, we were supposed to single someone out of the crowd of bystanders and give them a clear, identifiable task.  Something along the lines of, `You in the red shirt, call nine-one-one!'



And now that I'd fucked that up, I'd entrenched them.  The status quo was now quickly becoming `not listening to the supervillain', and it would be twice as hard to get them to go against the rest of the herd.



Which left me three unpleasant options.  The first option was that I could abandon that plan, look weak, and lose standing in the eyes of everyone present.  Alternately, I could speak up again, appeal to their humanity, beg, plead, demand, praying all the while for someone to come forward.  That was the second choice, and it would make me look even \emph{worse} to everyone watching, with only a miniscule chance of success.



The silence stretched on.  I knew it had only been five or six seconds, but it felt like a minute.



The third of my ugly options?  I could \emph{make} them listen.  Goad them into action with threats and violence.  It meant I risked provoking the same sort of chaos and violence I was hoping to combat, but I suspected that chance was relatively minor.  I could get people to do what I needed them to do.  I'd maybe earn their respect, but I'd probably earn their enmity at the same time.



Could I do this?  Could I become the bully, even if it was for the greater good?  I was going to hate myself for doing it, but I'd left my dad behind to be here.  I wasn't about to fail.



``Alright,'' I said, sounding calmer than I felt.  My fist clenched at my side.



I hesitated.  Someone was approaching.  I felt them passing through the bugs I'd dispersed through the crowd.  Charlotte.



``You're not wearing your mask,'' I said, the second she was close enough to hear me, my voice quiet.  ``Or the paper cube.''



``The cube got crushed when I was helping someone.  I was glad you didn't use your power,'' she said.  Then, loud enough that some people nearby could hear her, she asked me, ``What can I do?''



\emph{I owe her one hell of a favor.}



I'd had my bugs sweeping through nearby buildings since I'd arrived.  I hadn't really stopped, even after I got home.  I had found several of the wounded.  A man lying prone, two kids huddled near their mother.  The mother's face was sticky with blood, her breathing quick.  The children were bleeding too.  I could sense a man stumbling blindly through what had been his home, hands to his face.



I almost sent her after the blind man, but reconsidered.



I pointed at a warehouse, and spoke loud enough for others to hear, ``There's a woman and two little kids in there, you won't be able to help them alone.''  Which was a large part of why I had chosen them.



I spotted a twenty-something guy with an impressive bushy beard and no shirt.  Aside from one cut on his stomach and some smaller patches of shredded skin where the sand had caught him in the back, he seemed to be in okay shape.  ``You.  Help her.''



He looked at the older woman beside him.  His mother?  She was clearly hurt, and had the remains of two or three white t-shirts bundled around her arm.  It was clear the limb had been caught by the sand; it looked like a mummy's arm, only bloody.  Anticipating an excuse on his part, I pointing to the nearest group of injured and told him, ``They'll look after her.  There are people who need you more.  Second floor\emph{.  Go.}''



He looked at his mother, and the look she gave him was answer enough.  He helped her hobble over to the group of people I'd indicated, leaving her in their care, and joined Charlotte in running for the warehouse where the woman and kids were.



Now I just had to keep my momentum.



``You and your friend,'' I spoke to a middle-aged guy and his buddy.  ``There's a guy slowly bleeding out in the factory there.  Go help him.''



The second that passed before they moved to obey left my heart pounding.



I turned to the next person and stopped.  He was one of the few people with actual bandages on his wounds, and he stood near his family.  Even with the gauze pads strapped to his face, I recognized him from earlier.  Or, to be specific, I recognized the little boy R.J., and I knew this man as his father, patriarch of the rat infested house from early in the day.



``There's a blinded man in the brick building over there,'' I told him, facing him squarely.  ``Go help him.''



``Why?'' he challenged me, his voice gruff, his gaze hard.  ``I'm hurt.  If I go, I'm going to miss my turn with the ambulances.''



\emph{Asshole}.  There wasn't even a shred of gratitude for what I'd done to help him and his family, and he didn't even seem to need his turn at the ambulance that badly either.  I had to resist the urge to hit him or set my bugs on him.



Worse, I couldn't help but feel like he was seeing through the image I was trying to portray.  Seeing the \emph{girl} behind the mask, who was just trying to pretend she knew what she was doing.



I turned to the next person, a solidly built woman with scratches and the sandburns I was quickly coming to recognize all over her face.  She had even taped half of a sanitary pad over one eye.  It wasn't my brightest move, but I asked her, ``Are \emph{you} going to whine like a little girl, too, if I ask you to help someone?''



She smiled a little and shook her head.



``Good.  Go.  Left side of the building.  He's blind, and there's nobody else there to help.  I think he might have inhaled sand, he's coughing pretty violently.  Don't push him to move too fast or too much.  Take your time walking him back, if the bleeding isn't too severe.''



She obeyed, moving off with a powerful stride.  When I looked, R.J.'s dad was gone.  He was stomping off toward the ambulances, keeping the crowd between us, dragging his wife at his side with R.J. hurrying to keep up.  Knowing how angry he was, I had to hope he wasn't the type to take out his anger on his family.  I didn't want to be indirectly responsible for their pain.



There were more people to pick out of the crowd, more orders to give.  It was all about setting them up so that refusal made them look bad, both to themselves and to others.  Social pressure.



By the time I'd sent two more groups, some of the others were coming back to be directed to the next few injured.  I gave them their orders.



Which only raised the greater problem.  How were we supposed to handle these people who were hurt and waiting their turn?  They were scared and restless.  That unease bled over into their friends, families and maybe their neighbors, who were scared for themselves \emph{and} the people they cared about.  Already, they were gathering around the ambulances, pleading for help from too small a group of people, who had their hands full saving others' lives.  Some were simply asking the paramedics for advice while keeping a respectful distance, others were demanding assistance because they felt \emph{their} loved ones were more important than whoever was getting care or attention at that moment.  The paramedics couldn't answer everyone.



People in this area formed closely knit packs.  They would step up to defend the people they cared about far more quickly and easily than they had with my appeal to help strangers just minutes ago.  I didn't trust them to remain peaceful if this kept up.



What the hell was I supposed to do with them?



As lost as I felt in that moment, I managed to \emph{look} calm.  My bugs gave me an awareness of the situation, and my eyes swept over the scene to get a sense of the mood and what people were doing.



I spotted a mother picking at one of her son's wounds, and I realized what she was doing.  I hurried to stop her.  ``What are you doing?''



Riding the highs and the lows of emotion from the past hour or two, I might have come across sounding angrier than I was.  She quailed just a bit.



``He has glass in his arm.''



He did.  There were slivers of glass no longer than the nub of lead in an old-fashioned pencil, sticking out of his cuts.



``Those are probably okay to remove,'' I told her, ``But avoid disturbing any close to the arteries, here, here and here.''



``He doesn't have cuts there.''



``Good,'' I told her.  ``But you should know for later, for when you're helping others.''



She pointed at her leg.  Sand had flayed the skin of her foot and calf and turned the muscle a dirty brown color.  ``I can't really walk.''



``You won't need to.''



A plan was coalescing in my mind.  A way to give people something to do and give them some indication they'd eventually get help.  The problem was, I needed materials to carry this out, and there wasn't much nearby.  It meant I had to get the materials from my lair.  I wasn't willing to leave for any length of time, though, and I didn't want to spare Charlotte, either.



I had to use my bugs.  That wasn't so simple when the things I was retrieving weren't small.



I had a box of pens and markers in my room, for sketching out the costume designs.  I also had first aid kits in my bedside table upstairs and in the bathroom on the ground floor.  Bringing all of that stuff here meant opening the boxes and retrieving everything I needed, carting them here on a wave of crawling bugs, past puddles and flooded streets.



I collected markers, pens, bandages, ointments, iodine, candles and needles.  Especially needles.  Smaller bottles of hydrogen peroxide.  At least, I \emph{hoped} it was the iodine and hydrogen peroxide.  I couldn't exactly read the labels.  The bottle shapes felt right, anyways.



More people returned with the injured.  I administrated my bugs while I gave new directions to the rescue parties.



Just carrying the things on a tide of bugs wasn't going to work.  The crawling bugs couldn't pass through the water, and there was no way to have flying bugs carry things – too many of the objects were too heavy, even with the flying insects gathered on every inch of their surface and working in unison.



Minutes passed as I tried different configurations and formations of bugs, trying to wrangle things like the small bottle of hydrogen peroxide with my swarm.



Then I saw the woman with the maxi-pad eyepatch and a man of roughly the same age carting someone to the ambulance using a blanket attached to two broomsticks as a stretcher.



I could do the same thing.  I called on my black widow spiders, drawing some out from the terrariums where I had them contained.  Wasps carted them to the necessary spots, and I had them spin their silk around the objects in question and tie that silk to the necessary bugs.  Silk looped around the neck of a marker, then around a series of roaches, who could then be assisted by other bugs.  I did the same for the other things, the iodine, markers, pens, candles and more.



When I was done, I called the swarm to me.



I turned my attention to the injured who were clustering around the ambulances.



``Listen!'' I called out, using my bugs to augment my voice.  ``Some of you have been picking the glass out of your skin!  I understand it hurts, but you're slowing things down!''



I got some confused and angry looks.  I held up my hand to forestall any comments or argument.



``Any paramedic, nurse or doctor that helps you has to make absolutely sure that you don't have any glass embedded deep in your body.  I don't believe x-rays can detect glass-''



I paused as a paramedic snapped his head up to look at me.  Okay, so I was wrong.  I wished he hadn't reacted, though.  People were paying attention to the paramedics, they'd noticed, and it wasn't critical that these people know the exact details of the treatment they'd get.  If he'd just let me lie or be wrong, this would have gone smoother.



``Or at least, glass as fine as the shrapnel that hit you,'' I corrected myself.



A shrug and a nod from the paramedic.  I got my mental bearings and continued, ``If you're pulling the glass out of your cuts and wounds and you lose track of which ones you've tended to, they're going to have to explore the wounds to investigate, queue you up for x-rays and maybe even cut you open again later, after the skin has closed up, to get at any pieces they missed.''



I could see uneasy reactions from the crowd.  I raised my hand, just in time for the first of my swarm to arrive.  I closed my hand around a pen as the cloud of airborne insects delivered it to me.  They dispersed, and the pen remained behind.



``I'm going to give some of you pens and markers.  We're going to have a system to make all of this easier on the doctors.  Dotted lines around any injuries with glass sticking out.  Circles around wounds where the glass may be deeper.''



The paramedic waved me over.  I moved briskly through the crowd to the stretcher.



``Tetanus,'' he said, when I was close enough.  ``We need to know if they've had their shots.''



``They probably haven't,'' I replied, using my swarm to augment my voice, but not to carry it to the crowd.



``Probably not.  But we have to ask, and time we spend asking is time we could spend helping them.''



I grasped the hand of a grungy old man who stood next to me, stretching his arm out.  ``Have you had your shots?''



He shook his head.



I used the pen to draw a `T' on the back of his hand, circled it and drew a line through it.  I pressed the pen into the old man's hand, ``You go to people and ask them the same question.  If they haven't had their shots, draw the same thing.  If they have, just draw the T.''



I saw a glimmer of confusion in his eyes.  Was he illiterate?  I turned his hand over and drew a capital `T' on his palm.



``Like that, if they \emph{have} had their shots'' I said, raising his hand for people to see, then turned it around.  ``Like that if they haven't.''



He nodded and took the pen, turning to the not-quite-as-old man beside him.



I addressed the crowd, ``Remember, dotted line around the wounds if you can see the glass or if you're absolutely sure there's no glass in there, circle if you can't tell.  Once you or someone else has drawn the dotted line, you can take out the glass if it's smaller than your thumbnail.  If it's bigger, try to leave it alone!''



``We need some elbow room,'' the paramedic told me.  His blue gloves were slick with blood.  People were standing within two or three feet of him, watching what he was doing, trying to be close enough to be the next to get help when he was done with his current patient.



That wasn't the limit of the potential patients, either: there were the injured that Charlotte and the others were retrieving.  The people who hadn't been able to get here under their own power.



``We're changing locations,'' I called out.  I could see them reacting to that, balking at the idea.  ``If you're able to stand, it's going to be a \emph{long} time before you get the help you want.  There's plenty more people with worse injuries.  Suck it up!''



I waited for someone to challenge me on that.  Nobody did.



``If you listen and cooperate you'll get the help you want sooner.  We're going to gather inside the factory right here where we'll be clear of the worst of the dust.  It's dry inside, and there's enough space for all of us.''



It took some time for \emph{everyone} to get moving, but they did.  My bugs passed me some candles and a lighter and I started handing them out with the pens and markers.  I followed the mass of people into the defunct factory that was next to the ambulances.



Sheets and cloths were pulled from machinery and laid atop boxes and on the ground, so people had places to sit and lie down.  Gradually, people set about the process of marking the types of wounds and the presence of glass, buried or otherwise.



``Disinfectant?'' a woman asked me.



I turned.  She was older, in her mid-fifties, roughly my height, and she had a pinched face. ``What about it?''



``You've been pulling things out of the clouds of flies,'' she told me, ``Can you produce some disinfectant for us, or are you limited to art supplies and candles?''



I got the impression of a strict schoolteacher from her.  The kind who was a hardass with even the good students and a mortal enemy to the poor ones.



I reached out my hand, and a portion of my swarm passed over it.  Thanks to the fact that many of them were in contact with the bottle, it was easy enough to position my hand and know when to close it.  The bugs drifted away, and I was left holding the three-inch tall bottle.



My theatrics didn't seem to impress her.  Her tone was almost disparaging as she said, ``Nobody uses hydrogen peroxide anymore.  It delays recovery time.''



``That's not necessarily a bad thing,'' I said.  ``If the wounds heal over embedded glass, it'll be that much more unpleasant.''



``Do \emph{you} have medical training?'' she asked me, her tone disapproving.



``Not enough, no,'' I said with a sigh.  I had the swarm pass over my hand again, picking up the hydrogen peroxide and depositing another plastic bottle in its place.  ``Iodine?''



``Thank you,'' she said, in a tone that was more impatient than grateful.  ``We're going to need more than this.''



``I'll see what I can do,'' I told her, trying not to sound exasperated.



She headed for a group of people and knelt by one of the wounded who was lying on a sheet.  I could see her posture and expression soften as she talked to them.  So she wasn't like that with other people.



Whatever.  I'd been prepared to be hated when I committed to villainy.



I gathered all of the supplies I'd brought and sent more bugs out to scout for more.



What I wouldn't give for a working cell phone, to find out about how Tattletale was doing, even to ask after my dad.  But cell phones had computer chips, and computer chips had silicon.



Everything that was electronic and more complicated than a toaster was probably fried, with exceptions for some tinker-made stuff.



There was no use dwelling on the fact that two people I cared about were gravely hurt.  I couldn't do anything about it now, and time spent wondering was time I wasn't protecting and helping \emph{these} people.



In terms of protecting these people, I spread my bugs out over every surface, until a potential threat wouldn't be able to take a step without killing one.  It would serve as advance warning in case any members of Hookwolf's alliance came through to make trouble.  I spread out some flying insects to try to detect airborne threats like Rune.



Most of the flying bugs, however, I was using to sweep over my surroundings, checking buildings and building interiors.  I wanted first aid kits, anything these people could use to clean their wounds.  Noting the lack of suture threads, I had my spiders start using their silk to spin something long, thick and tough enough, threading it through the holes of needles for their use.



It would slow down my costume production a touch, but I could deal.



``That doesn't look very sterile,'' a woman said, from behind me, as I checked the length of the thread one set of spiders had produced.  It was the pinched, gray-haired woman from just a little bit ago.



``More than you'd think.  I raised these little ladies myself.  They live in terrariums.''



``That doesn't mean it's clean enough to thread through someone's open wounds.''



``No,'' I replied, feeling a bit irritated, ``But in the absence of good alternatives, I'd rather use this and then supply everyone here with antibiotics at some point in the next day or so.  Which they probably need anyways.''



``People use antibiotics too often,'' she said.  ``I try to make a point of using them sparingly in my clinic.''



\emph{Seriously}?  ``I think situations like this are the \emph{exact} right time to use antibiotics.  These people have open wounds, they're undernourished, dehydrated, stressed, their immune systems are probably shot, their environments are filthy, there's probably countless other reasons.''



She said something, sounding even more irritated than before.  I think it was a repeat of the question from earlier, about my credentials in medicine.  I wasn't listening.



The paramedics hadn't come out of the ambulance in several minutes.  A check with my bugs found them lying on the floor of the ambulance.  No blood, as far as I could tell.



Ignoring the woman, I turned and headed for the door, hurrying outside.  She barked something snide at my back.



I was battle ready as I approached the ambulance and checked the area.  Nobody.



Stepping inside, I checked on the paramedics and the patient with an oxygen balloon strapped to his face.  The paramedics were beyond help, dead, their heads twisted at an ugly angle.  The patient hadn't been dispatched the same way.  I checked his throat to find him still warm, but he wasn't breathing and he had no pulse.  I squeezed the balloon, and huge amounts of blood bubbled from what I had taken to be a shallow cut in his chest. The bubbles meant the oxygen was leaking from his punctured lung.



This wound – there was no way he could have had it when he came into the ambulance.  It was fresh.  All three of the people here had been executed.  It had been done in cold blood, clean, and I hadn't even noticed with my bugs on watch.



Which left me \emph{very} concerned for the people I'd left in the warehouse.  I hopped down from the back of the ambulance, checked my surroundings, and then ran across the street.



I was a single step inside the door when I saw him.  Tall, faceless, featureless, but for the chains and ball joints that connected his ceramic-encased limbs.  One hand was raised, a single finger raised, ticking from side to side like a metronome.  Like an old-fashioned parent scolding an errant child.



The other hand was folded back, a long telescoping blade extended from the base of Mannequin's palm.  The blade was pressed to the neck of the gray-haired doctor, so she had to stand on her tiptoes, her head pressed back against his chest.



I didn't have a chance to move, to speak, or to use my power before he retracted the blade.  It slid across her throat, shearing through the skin, and arterial blood sprayed forth to cover some of the ground between us.  She collapsed to the ground.



Mannequin's knife hand went limp, dangling at his side.  His other hand remained in position, finger wagging, as if admonishing me for what I had been doing.  Saving people from the Nine, tending to the hurt and scared.



I should have anticipated this.



I stepped forward, almost without thinking about it, and he dropped his other hand while taking three long steps to back away from me. His movements were ungainly, as if he was about to collapse to the ground with each one.  No sooner had I wondered why when I saw his feet.  His `toes' pointed at the ground, and blades had sprouted from slots at the front of each foot.  He was perched precariously on the honed knife points, walking on the blades.



Reaching behind my back, I drew my baton and knife.  I tensed as he moved in reaction, closing half the distance between us, lurching three or four feet to the right, then back again.



I caught on immediately.  He was evading the bugs that had been hovering in the air between us, the knife-stilts that extended from his feet delicately avoiding contact with the bugs that were on the ground.  The contact he \emph{did} make with the bugs was gentle, sliding against them like a brush of wind.  I only noticed because I was paying attention.



He didn't \emph{need} to avoid my swarm.  He was taunting me.  Letting me know exactly how he had gotten so close without me realizing it.



I flicked out my baton to its full length.  He responded by doing the same with the telescoping blades that unfolded from his arms.  His weapons were longer, both sharp.



Not taking my eyes off him, I used my bugs and my peripheral vision to track the other people in the warehouse.  Too many were too hurt to move, and those who \emph{could} move had backed into corners and to places where they had cover.



Still, this was \emph{his} battlefield.  He had far too many hostages at his disposal.  He was faster than me, stronger, tougher.



I was pretty damn sure that his power was as complete a counter to mine as anyone could hope for.  Anyone who had paid attention to the news in the past five years knew who he was, what his story was.  Mannequin had once been a tinker who specialized in biospheres, terrariums and self-contained ecosystems.  A tinker who specialized in sustaining life, sheltering it from outside forces; forces that included water, weather, space\ldots and \emph{bugs}.



The only difference between then and now was that he was using his power to help and protect himself and himself only.



``Motherfucker.''  Even without intending to do it, I used my swarm to carry my voice.  His head craned around, as if to look at the swarming bugs who had just, for all intents and purposes, spoken.  Eventually his `face' turned back to me.



``I have no idea how the fuck I'm going to do it,'' my voice was a low snarl, barely recognizable as my own beneath my anger and the noises of the swarm.  ``But I'm going to make you regret that.''





